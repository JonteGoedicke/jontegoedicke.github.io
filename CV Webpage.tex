\documentclass[10pt, letterpaper]{article}

% Packages:
\usepackage[
    ignoreheadfoot, % set margins without considering header and footer
    top=2 cm, % seperation between body and page edge from the top
    bottom=2 cm, % seperation between body and page edge from the bottom
    left=2 cm, % seperation between body and page edge from the left
    right=2 cm, % seperation between body and page edge from the right
    footskip=1.0 cm, % seperation between body and footer
    % showframe % for debugging 
]{geometry} % for adjusting page geometry
\usepackage[explicit]{titlesec} % for customizing section titles
\usepackage{tabularx} % for making tables with fixed width columns
\usepackage{array} % tabularx requires this
\usepackage[dvipsnames]{xcolor} % for coloring text
\definecolor{primaryColor}{RGB}{0, 79, 144} % define primary color
\usepackage{enumitem} % for customizing lists
\usepackage{fontawesome5} % for using icons
\usepackage{amsmath} % for math
\usepackage[
    pdftitle={Jonte Gödicke CV},
    pdfauthor={Jont Gödicke},
    colorlinks=true,
    urlcolor=primaryColor
]{hyperref} % for links, metadata and bookmarks
\usepackage[pscoord]{eso-pic} % for floating text on the page
\usepackage{calc} % for calculating lengths
\usepackage{bookmark} % for bookmarks
\usepackage{changepage} % for one column entries (adjustwidth environment)
\usepackage{paracol} % for two and three column entries
\usepackage{ifthen} % for conditional statements
\usepackage{needspace} % for avoiding page brake right after the section title
\usepackage{iftex} % check if engine is pdflatex, xetex or luatex

% Ensure that generate pdf is machine readable/ATS parsable:
\ifPDFTeX
    \input{glyphtounicode}
    \pdfgentounicode=1
    \usepackage[T1]{fontenc}
    \usepackage[utf8]{inputenc}
    \usepackage{lmodern}
\fi

\usepackage[default, type1]{sourcesanspro} 

% Some settings:
\AtBeginEnvironment{adjustwidth}{\partopsep0pt} % remove space before adjustwidth environment
\pagestyle{empty} % no header or footer
\setcounter{secnumdepth}{0} % no section numbering
\setlength{\parindent}{0pt} % no indentation
\setlength{\topskip}{0pt} % no top skip
\setlength{\columnsep}{0.15cm} % set column seperation
\makeatletter
\let\ps@customFooterStyle\ps@plain % Copy the plain style to customFooterStyle
\makeatother
\pagestyle{customFooterStyle}

\titleformat{\section}{
    % avoid page braking right after the section title
    \needspace{4\baselineskip}
    % make the font size of the section title large and color it with the primary color
    \Large\color{primaryColor}
}{
}{
}{
    % print bold title, give 0.15 cm space and draw a line of 0.8 pt thickness
    % from the end of the title to the end of the body
    \textbf{#1}\hspace{0.15cm}\titlerule[0.8pt]\hspace{-0.1cm}
}[] % section title formatting

\titlespacing{\section}{
    % left space:
    -1pt
}{
    % top space:
    0.3 cm
}{
    % bottom space:
    0.2 cm
} % section title spacing

% \renewcommand\labelitemi{$\vcenter{\hbox{\small$\bullet$}}$} % custom bullet points
\newenvironment{highlights}{
    \begin{itemize}[
        topsep=0.10 cm,
        parsep=0.10 cm,
        partopsep=0pt,
        itemsep=0pt,
        leftmargin=0.4 cm + 10pt
    ]
}{
    \end{itemize}
} % new environment for highlights

\newenvironment{highlightsforbulletentries}{
    \begin{itemize}[
        topsep=0.10 cm,
        parsep=0.10 cm,
        partopsep=0pt,
        itemsep=0pt,
        leftmargin=10pt
    ]
}{
    \end{itemize}
} % new environment for highlights for bullet entries


\newenvironment{onecolentry}{
    \begin{adjustwidth}{
        0.2 cm + 0.00001 cm
    }{
        0.2 cm + 0.00001 cm
    }
}{
    \end{adjustwidth}
} % new environment for one column entries

\newenvironment{twocolentry}[2][]{
    \onecolentry
    \def\secondColumn{#2}
    \setcolumnwidth{\fill, 4.5 cm}
    \begin{paracol}{2}
}{
    \switchcolumn \raggedleft \secondColumn
    \end{paracol}
    \endonecolentry
} % new environment for two column entries

\newenvironment{threecolentry}[3][]{
    \onecolentry
    \def\thirdColumn{#3}
    \setcolumnwidth{1 cm, \fill, 4.5 cm}
    \begin{paracol}{3}
    {\raggedright #2} \switchcolumn
}{
    \switchcolumn \raggedleft \thirdColumn
    \end{paracol}
    \endonecolentry
} % new environment for three column entries

\newenvironment{header}{
    \setlength{\topsep}{0pt}\par\kern\topsep\centering\color{primaryColor}\linespread{1.5}
}{
    \par\kern\topsep
} % new environment for the header

\newcommand{\placelastupdatedtext}{% \placetextbox{<horizontal pos>}{<vertical pos>}{<stuff>}
  \AddToShipoutPictureFG*{% Add <stuff> to current page foreground
    \put(
        \LenToUnit{\paperwidth-2 cm-0.2 cm+0.05cm},
        \LenToUnit{\paperheight-1.0 cm}
    ){\vtop{{\null}\makebox[0pt][c]{
        \small\color{gray}\textit{Last updated in October 2024}\hspace{\widthof{Last updated in October 2024}}
    }}}%
  }%
}%

% save the original href command in a new command:
\let\hrefWithoutArrow\href

% new command for external links:
\renewcommand{\href}[2]{\hrefWithoutArrow{#1}{\mbox{\ifthenelse{\equal{#2}{}}{ }{#2 }\raisebox{.15ex}{\footnotesize \faExternalLink*}}}}


\begin{document}
\placelastupdatedtext
    \begin{header}
        \fontsize{30 pt}{30 pt}
        \textbf{Jonte Gödicke}

        \vspace{0.3 cm}

        \normalsize
        \mbox{{\footnotesize\faMapMarker*}\hspace*{0.13cm}  Office: 218, Vivatgasse 7, Bonn}
        \kern 0.5 cm
        \mbox{\hrefWithoutArrow{mailto:jonte.goedicke@uni-hamburg.de}{{\footnotesize\faEnvelope[regular]}\hspace*{0.13cm}math@jonte-goedicke.com}}
        \kern 0.5 cm
        \mbox{\hrefWithoutArrow{tel:+49 40 42838-5189}{{\footnotesize\faPhone*}\hspace*{0.13cm}+49 40 42838-5189}}
        \kern 0.5 cm
        \mbox{\hrefWithoutArrow{https://jonte-goedicke.com/}{{\footnotesize\faLink}\hspace*{0.13cm}https://jonte-goedicke.com/}}
        \kern 0.5 cm
        \mbox{\hrefWithoutArrow{https://github.com/JonteGoedicke}{{\footnotesize\faGithub}\hspace*{0.13cm}JonteGoedicke}}
        \kern 0.5 cm
    \end{header}

    \vspace{0.3 cm- 0.3cm}

    
    \section{Education}



        
        \begin{twocolentry}{
         October 2021- August 2025

        }
            \textbf{PhD in Mathematics}, \textit{University of Hamburg, Germany}
            \begin{highlights}
                \item Cluster of Excellence Quantum Universe
                \item Adviser: Tobias Dyckerhoff
                \item Grade: 1.0 Magna Cum Laude
            \end{highlights}
        \end{twocolentry}


        \vspace{0.2 cm}

        \begin{twocolentry}{

        2019-2021
        
        }
            \textbf{Master of Science in Mathematical Physics}, \textit{University of Hamburg, Germany}
            \begin{highlights}
                \item Master thesis: Examples of derived formal moduli problems
                \item Adviser: Julian Holstein 
                \item Grade: 1.05
            \end{highlights}
        \end{twocolentry}


        \vspace{0.2 cm}

        \begin{twocolentry}{

        2015-2019 
            
        }
            \textbf{Bachelor of Science in Physics}, \textit{University of Hamburg, Germany}
            \begin{highlights}
                \item Bachelor Thesis: Various topics in classical, bosonic, semiclosed string theory 
                \item Advisor: Marco Zagermann 
                \item Grade: 1.04
            \end{highlights}
        \end{twocolentry}


        \vspace{0.2 cm}

        \begin{twocolentry}{

    2007-2015
           
        }
            \textbf{Abitur}, \textit{Gymnasium Osterholz-Scharmbeck, Germany}
            \begin{highlights}
                \item Grade: 1.1
            \end{highlights}
        \end{twocolentry}

\section{References}

\begin{onecolentry}
                \textbf{Tobias Dyckerhoff}, University of Hamburg \\
                Mail: tobias.dyckerhoff@uni-hamburg.de 

\end{onecolentry}

\vspace{0.2cm}

\begin{onecolentry}
    \textbf{Joachim Kock}, Universitat Aut\`{o}noma de Barcelona \\
    Mail: joachim.kock@uab.cat
\end{onecolentry}

\vspace{0.2cm}

\begin{onecolentry}
    \textbf{Julian Holstein}, University of Hamburg \\
    Mail: julian.holstein@uni-hamburg.de 
\end{onecolentry}

\vspace{0.2cm}

\begin{onecolentry}
    \textbf{Paul Wedrich}, University of Hamburg \\
    Mail: paul.wedrich@uni-hamburg.de 
\end{onecolentry}


\section{Employement}

\begin{twocolentry}
    {Oktober 2025-2027}
    {\textbf{Max-Planck-Institut for Mathematics Bonn}, Postdoc, Mentored by Catharina Stroppel}
\end{twocolentry}

\vspace{0.2cm}

\begin{twocolentry}{

    Oktober 2024 - 2025
           
        }
            \textbf{University of Hamburg}, University research assistant, after finishing the PhD as a Postdoc associated with the CRC 1624 "Higher Structures, Moduli Spaces and Integrability"
        \end{twocolentry}


\section{Preprints}

 \begin{onecolentry}
                \textbf{An $\infty$-category of $2$-Segal Spaces}

                \vspace{0.10 cm}

                \mbox{\textbf{Jonte Gödicke}}, \url{https://arxiv.org/abs/2407.13357}

                \vspace{0.10cm}
                
                \textbf{Content}: We provide an equivalence between the $\infty$-category of algebra objects in $\infty$-categories of spans with a subcategory of the $\infty$-category of spans of $2$-Segal objects.
                \vspace{0.10 cm}

\end{onecolentry}
\vspace{0.2cm}

\begin{onecolentry}
                \textbf{Simons Lectures on Categorical Symmetries,\\
                Lecture Notes: Applied Cobordism Hypothesis, Lecturer David Jordan}

                \vspace{0.10 cm}

                \mbox{Aaron Hofer, Anja $\Check{S}$vraka and \textbf{Jonte Gödicke}}, \url{https://arxiv.org/abs/2411.09082}
                
                \vspace{0.10cm}
                
                \textbf{Content}: We provide lecture notes for the lecture "Applied Cobordism Hypothesis" given by David Jordan during the Global Categorical Symmetries School in Les Diableret in September 2023.
                \vspace{0.10 cm}




\end{onecolentry}


\section{Work in Progress}

 \begin{onecolentry}
                \textbf{Rigid $2$-Segal spaces}

                \vspace{0.10 cm}

                \mbox{\textbf{Jonte Gödicke}}, \url{https://www.birs.ca/events/2024/5-day-workshops/24w5266/videos}

                \vspace{0.10 cm}
                
                \textbf{Content}: We use $(\infty,2)$-categorical techniques to classify those $2$-Segal spaces that give rise to (derived) multi-fusion categories.
                \vspace{0.10 cm}

\end{onecolentry}



    
\section{Conference Talks}

     \begin{twocolentry}{

    January 2024
           
        }
            \textbf{Workshop Higher Segal Spaces and their Applications to Algebraic K-Theory, Hall Algebras, and Combinatorics}, \\
            \textit{Banff International Research Station}, Title: Rigid $2$-Segal Spaces
            
           
        \end{twocolentry}
      \vspace{0.2 cm}
        \begin{twocolentry}{

    September 2023
           
        }
            \textbf{Global Categorical Symmetries School}, \\
            \textit{Les Diableret, Switzerland}, Gong Show Talk, Title: Towards Derived Turaev-Viro Theory 
            
           
        \end{twocolentry}

    \vspace{0.2cm}

     \begin{twocolentry}{

    October 2022
           
        }
            \textbf{Summer School: Donaldson Thomas Invariants in Derived Symplectic Geometry}, \\
            \textit{Aussois, France},  Title: Motivic DT Invariants after Kontsevich Soibelmann
            
           
        \end{twocolentry}
        
 \vspace{0.2cm}

 \begin{twocolentry}{

    September 2021
           
        }
            \textbf{Summer School: Irregular Riemann Hilbert Correspondences}, \\
            \textit{Aussois, France},  Title: Local Moduli for marked meromorphic flat bundles
            
           
        \end{twocolentry}
        
 \vspace{0.2cm}

    
    \section{Seminar Talks}

     \begin{twocolentry}{October 2025}
    \textbf{MPIM Topology Seminar},\\
    \textit{Bonn}, Title: The universal Hecke algebroid
    \end{twocolentry}


    \begin{twocolentry}{May 2025}
    \textbf{QTCat Seminar},\\
    \textit{Hamburg}, Title: The universal braided monoidal Hecke category
    \end{twocolentry}

    \vspace{0.2cm}

    \begin{twocolentry}{March 2025}
    \textbf{Algebra and Geometry Seminar},\\
    \textit{Hong Kong}, Title: Multi-fusion categories from 2-Segal spaces
    \end{twocolentry}

    \vspace{0.2cm}
    
    \begin{twocolentry}
        {December 2024}
        \textbf{Quantum Mathematics Research Seminar},\\
        \textit{Odense, Denmark}, Title: Multi-fusion categories from 2-Segal spaces
    \end{twocolentry}

    \vspace{0.2cm}
        
        \begin{twocolentry}{

    May 2024
           
        }
            \textbf{Higher Structures in Quantum Field Theory Seminar}, \\
            \textit{Munich, Germany},  Title: Rigid Hall monoidal structures
            
           
        \end{twocolentry}
 \vspace{0.2cm}

\begin{twocolentry}{

    May 2024
           
        }
            \textbf{Algebra and Topology Seminar}, \\
            \textit{Kopenhagen, Denmark},  Title: Which 2-Segal objects are rigid?
            
           
        \end{twocolentry}
 
\vspace{0.2cm}

 \begin{twocolentry}{

    November 2023
           
        }
            \textbf{Higher Structure Seminar}, \\
            \textit{Hamburg, Germany},  Title: Rigid 2-Segal spaces
            
           
        \end{twocolentry}
        
 \vspace{0.2cm}
    
 \begin{twocolentry}{

    November 2023
           
        }
            \textbf{Higher Structure Seminar}, \\
            \textit{Hamburg, Germany},  Title: Introduction to the Cobordism Hypothesis
            
           
        \end{twocolentry}
        
 \vspace{0.2cm}

  \begin{twocolentry}{

    May 2023
           
        }
            \textbf{Higher Structure Seminar}, \\
            \textit{Hamburg, Germany},  Title: Symplectic Cohomology
            
           
        \end{twocolentry}
        
 \vspace{0.2cm}

 \begin{twocolentry}{

    January 2022
           
        }
            \textbf{ZMP Seminar}, \\
            \textit{Hamburg, Germany},  Title: Categorification of Cluster algebras, with Merlin Christ
            
           
        \end{twocolentry}
        
 \vspace{0.2cm}

  \begin{twocolentry}{

    April 2021
           
        }
            \textbf{Higher Structure Seminar}, \\
            \textit{Hamburg, Germany},  Title: Classical Hall Algebras
            
           
        \end{twocolentry}
        
 \vspace{0.2cm}

 \begin{twocolentry}{

    May 2020
           
        }
            \textbf{Higher Structure Seminar}, \\
            \textit{Hamburg, Germany},  Title: Algebraic Stacks
            
           
        \end{twocolentry}
        
 \vspace{0.2cm}

  \begin{twocolentry}{

    December 2019
           
        }
            \textbf{Higher Structure Seminar}, \\
            \textit{Hamburg, Germany},  Title: Deformation of Singularities 
            
           
        \end{twocolentry}
        
 \vspace{0.2cm}

\section{Organized Events}

\begin{twocolentry}{

    Fall 2023
           
        }
            \textbf{Live Stream of the lecture by Prof. Dr. Denis-Charles Cisinski about Formalization of Higher Category Theory}, \\
            \textit{Hamburg, Germany} Organized with Jonas Linssen and Markus Zetto
            
           
        \end{twocolentry}
        
 \vspace{0.2cm}

  \begin{twocolentry}{

    March 2023
           
        }
            \textbf{Mini Workshop on Higher Categorical Methods in Algebra and Geometry}, \\
            \textit{Hamburg, Germany} Co-Organizer with Tobias Dyckerhoff, Matthew Habermann, and Angus Rush
            
           
        \end{twocolentry}
        
 \vspace{0.2cm} 

\begin{twocolentry}{

    April 2022 - Fall 2023
           
        }
            \textbf{PhD Pizza Seminar}, \\
            \textit{Hamburg, Germany} Organized with Angus Rush, David Jaklitsch, Thomas Stempfhuber and Aaron Hofer
            
           
        \end{twocolentry}
        
 \vspace{0.2cm}


 
\section{Teaching Experience}

\begin{twocolentry}
    {Spring 2025}
    Tutor Linear Algebra 2
    
\end{twocolentry}

\vspace{0.2cm}

\begin{twocolentry}{

    Fall 2024
           
        }
           Tutor for Linear Algebra 1
            
           
        \end{twocolentry}
        
 \vspace{0.2cm}

 \begin{twocolentry}{

    Fall 2022
           
        }
           Tutor for Linear Algebra 1
            
           
        \end{twocolentry}
        
 \vspace{0.2cm}

 \begin{twocolentry}{

    Spring 2022
           
        }
         Tutor for Algebraic Topology 2
            
           
        \end{twocolentry}
        
 \vspace{0.2cm}

 \begin{twocolentry}{

    Fall 2021
           
        }
           Tutor for Mathematics for Physicists 3 
            
           
        \end{twocolentry}
        
 \vspace{0.2cm}

 \begin{twocolentry}{

    Spring 2021
           
        }
           Student Tutor for Mathematics for Physicists 4
            
           
        \end{twocolentry}
        
 \vspace{0.2cm}

  \begin{twocolentry}{

    Fall 2020
           
        }
           Student Tutor for Algebra 
            
           
        \end{twocolentry}
        
 \vspace{0.2cm}

 \begin{twocolentry}{

    Spring 2020
           
        }
           Student Tutor for Mathematics for Physicists 4
            
           
        \end{twocolentry}
        
 \vspace{0.2cm}

  \begin{twocolentry}{

    Fall 2019
           
        }
           Student Tutor for Mathematics for Physicists 3
            
           
        \end{twocolentry}
        
 \vspace{0.2cm}

  \begin{twocolentry}{

    Spring 2019
           
        }
           Student Tutor for Mathematics for Physicists 4
            
           
        \end{twocolentry}
        
 \vspace{0.2cm}

  \begin{twocolentry}{

    Fall 2018
           
        }
           Student Tutor for Mathematics for Physicists 3
            
           
        \end{twocolentry}
        
 \vspace{0.2cm}

 \begin{twocolentry}{

    Spring 2018
           
        }
           Student Tutor for Mathematics for Physicists 2
            
           
        \end{twocolentry}
        
 \vspace{0.2cm}

  \begin{twocolentry}{

    Spring 2018
           
        }
           Student Tutor for Theoretical Physics 2
            
           
        \end{twocolentry}
        
 \vspace{0.2cm}

  \begin{twocolentry}{

    Fall 2017
           
        }
           Student Tutor for Mathematics for Physicists 1
            
           
        \end{twocolentry}
        
 \vspace{0.2cm}

 
  \begin{twocolentry}{

    Fall 2017
           
        }
           Student Tutor for Theoretical Physics 1
            
           
        \end{twocolentry}
        
 \vspace{0.2cm}

 
  \begin{twocolentry}{

    Spring 2017
           
        }
           Student Tutor for Experimental Physics 2
            
           
        \end{twocolentry}
        
 \vspace{0.2cm}


\section{Soft Skill Courses}

\begin{twocolentry}
{July 2024}
    \textbf{Hamburg Research Academy:} How to Advance Your Research Career and Get Funding 
\end{twocolentry}

\vspace*{0.2cm}


 \begin{twocolentry}
 {April-June 2024}

      \textbf{PIER Desy} Academic Writing Course

 \end{twocolentry}
    







 \section{Further attended Mathematical Events}

 \begin{twocolentry}{

    August 2023
           
        }
            \textbf{Conference Higher Structures in Functorial Field Theories}, \\
            \textit{Regensburg, Germany}
            
           
        \end{twocolentry}
                
 \vspace{0.2cm}

\begin{twocolentry}{

    June 2023
           
        }
            \textbf{Hausdorff School: TQFT and their connections to Representation theory and Mathematical Physics}, \\
            \textit{Bonn, Germany}
            
           
        \end{twocolentry}
        
 \vspace{0.2cm}

 \begin{twocolentry}{

    May 2022
           
        }
            \textbf{School: Finiteness Conjecture for Skein modules}, \\
            \textit{Matemale, France}
            
           
        \end{twocolentry}
        
 \vspace{0.2cm}

\begin{twocolentry}{

    Januar-May 2021
           
        }
            \textbf{Online Seminar: Champs et Homotopie en G\'{e}oem\'{e}trie alg\'{e}brique}, \\
            \textit{Toulouse, France}
            
           
        \end{twocolentry}
        
 \vspace{0.2cm}

 \begin{twocolentry}{

    April 2018
           
        }
            \textbf{String Theory Steilkurs}, \\
            \textit{Hamburg, Germany}
            
           
        \end{twocolentry}
        
 \vspace{0.2cm}

\section{Language}
        
        \begin{onecolentry}
            German Native speaker
        \end{onecolentry}

        \vspace{0.2 cm}

        \begin{onecolentry}
           English, Fluently
        \end{onecolentry}

        \vspace{0.2 cm}

        \begin{onecolentry}
           Latin, Advanced certificate
        \end{onecolentry}

        

 
\end{document}